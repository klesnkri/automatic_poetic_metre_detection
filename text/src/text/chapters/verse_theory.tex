\chapter{Verse theory}\label{chap:verse-theory}

\begin{chapterabstract}
This chapter introduces essential concepts from verse theory that are used throughout this thesis and are vital for a better understanding of the rest of this work.
\end{chapterabstract}

\section{Poem structure}

\subsection{Verse}
Poetry is written in verse. A common misconception is that one verse is equivalent to one line in a poem. That is not true in all cases. The verse does not have to end with a line end, but it can be spread across multiple lines. This often occurs, for example, inside dialogues in verse dramas. When a verse does not end with a line end, it is called an \emph{enjambement}.~\cite{TeorieLiteraturySS}

For an example of an enjambement, see Figure~\ref{fig:enjambement}.

\begin{figure}[htpb]
    \centering
    \settowidth{\versewidth}{se těsně k ní, vytáhne z haleny}
    \begin{verse}[\versewidth]
    Dí paní domu; dítě přiblíží\\
    se těsně k ní, vytáhne z haleny\\
    list složený...
    \end{verse}
    \caption[Enjambement]{Enjambement~\cite{UvodTeorieVerse}}\label{fig:enjambement}
\end{figure}

\subsection{Strophe}
Verses inside a poem can be organised into strophes. A strophe is a group of verses that form a semantic unit and tend to be graphically separated. The strophe is then repeated throughout the poem with similar or almost similar properties (same number of verses, same metrical or rhyme scheme). Strophes tend to contain 2--14 verses -- strophes with an even number of verses are more common. When shorter and longer verses alternate in a strophe, then usually in such an order that a longer verse comes before a shorter one. The last verse of a strophe tends to be shorter or less rhythmically regular.~\cite{TeorieLiteraturySS} Some strophes even have special names assigned to them, for example, \emph{the Sapphic stanza}, \emph{the Alcaic stanza}, \emph{the Second Asclepiad stanza}, or \emph{the Fourth Asclepiad stanza}.~\cite{UvodTeorieVerse}

\subsection{Paragraph (Stanza)}
However, in many poems, groups of verses can be encountered that are graphically separated; nevertheless, their internal organisation lacks any regularity. These are not called strophes but paragraphs (stanzas).~\cite{TeorieLiteraturySS}

\section{Versification systems}
This section introduces four important versification systems -- \emph{syllabic}, \emph{quantitative}, \emph{tonic}, and \emph{syllabotonic} -- along with the different approaches towards versification. However, only syllabotonic versification will be further discussed in the rest of this work, as within the Corpus of Czech Verse, metres are assigned only to syllabotonic verses.~\cite{CorpusCzechVerse}

\subsection{Syllabic versification system}
Probably the oldest versification system used in Indo-European languages, syllabic, distinguishes verses only by the number of syllables they contain. Syllabic versification does not care whether the syllables are accented or long, the only important thing is their number.~\cite{UvodTeorieVerse} In Czech poetry, the syllabic verse was used until the end of the 18\textsuperscript{th} century. After that, it appeared only episodically, especially in folk poetry.~\cite{TeorieLiteraturySS}

In the excerpt from the poem \emph{Co Bůh? Člověk?} by Fridrich Bridel (see Figure~\ref{fig:syllabic-versification}), no regularity can be found in the alternation of long and short syllables or the accented and non-accented syllables. However, it can be noted that 7-syllable and 8-syllable verses regularly alternate and verses with the same number of syllables rhyme. Therefore, it represents a syllabic versification.~\cite{TeorieLiteraturySS}

\begin{figure}[htpb]
    \centering
    \settowidth{\versewidth}{stro|jí | se | vše|cko | k lí|tá|ní. \textsubscript{(8 syll.)}}
    \begin{verse}[\versewidth]
    Ja|ký | boj? | Ja|ké | hnu|tí? \textsubscript{(7 syll.)}\\
    mně | vstu|pu|jí | na | myš|le|ní? \textsubscript{(8 syll.)}\\
    Mám|-li | snad | za|hy|nou|ti? \textsubscript{(7 syll.)}\\
    Či|ji | mdlé | při|ro|ze|ní. \textsubscript{(7 syll.)}\\
    Což | to? | Věc | vel|mi | rych|lá, \textsubscript{(7 syll.)}\\
    a|neb | jest|-li | ňá|ké | zdá|ní, \textsubscript{(8 syll.)}\\
    ros|tou | mně | ja|kás | kří|dla, \textsubscript{(7 syll.)}\\
    stro|jí | se | vše|cko | k lí|tá|ní. \textsubscript{(8 syll.)}\\
    \end{verse}
    \caption{Syllabic versification}\label{fig:syllabic-versification}
\end{figure}

\subsection{Quantitative versification system}
The quantitative versification system differentiates between long and short syllables. The long and short syllables are annotated according to the rules that take diphthongs, vocals, and syllable-forming consonants into account. Quantitative verse can be found in Greek and Roman poetry. In Czech poetry, it was used mainly in the 16\textsuperscript{th} century and then shortly in the 1920s.

In the quantitative versification system, a syllable can be long by nature or long by position. Syllable long by nature contains a long vocal or a diphthong. Meanwhile, a syllable long by position contains either a short vocal or a syllable-forming consonant \emph{l} or \emph{r}. This short vocal or syllable-forming consonant is followed by two or more consonants (not necessarily belonging to the same syllable). On the other hand, when a syllable contains a short vocal or a syllable-forming consonant followed by only one consonant, the syllable is classified as short.

When a syllable contains a short vocal or a syllable-forming consonant followed by exactly two consonants and one of the two consonants is \emph{l}, \emph{r}, \emph{ř}, \emph{m} or \emph{n}, the syllable can be long or short depending on the context.

In the passage from the poem \emph{Noční bdění} by I. V. Šimko (see Figure~\ref{fig:quantitative-versification}), all long syllables occupy an odd position -- except for the second position in the last verse and the final positions in all verses except the last one. The final position in verse represented a common exception and could be occupied by both long and short syllables. Therefore, the poem is classified as quantitative.~\cite{TeorieLiteraturySS}

\begin{figure}[htpb]
    \centering
    \settowidth{\versewidth}{\uline{a} | v le|\uline{sích} | zpě|\uline{vák} | mi|\uline{lost}|\uline{ný}}
    \begin{verse}[\versewidth]
    \uline{sen} | mi|\uline{lý} | po|\uline{koj}|ně | \uline{lí}|\uline{tá},\\
    \uline{a} | v le|\uline{sích} | zpě|\uline{vák} | mi|\uline{lost}|\uline{ný}\\
    \uline{s pří}|ro|\uline{dou} | ce|\uline{lou} | spo|\uline{čí}|\uline{vá}:\\
    \uline{teh}|dy | \uline{já} | se | \uline{mar}|ně | \uline{trá}|\uline{pím}\\
    \uline{blou}|\uline{dě} | \uline{v há}|ji | \uline{až} | do | \uline{rá}|na\\
    \end{verse}
    \caption[Quantitative versification]{Quantitative versification (long syllables are underlined)}\label{fig:quantitative-versification}
\end{figure}

\subsection{Tonic versification system}
Tonic versification was the second most important versification system in medieval Europe. The tonic verse normalises the number of accents in a line. Usage in Czech poetry is very rare, oftentimes readers confuse it with free verse~\cite{UvodTeorieVerse} (verse without a metrical norm~\cite{TeorieLiteraturySS}).

In the example of a tonic poetic text (see Figure~\ref{fig:tonic-versification}), every verse has exactly four accents, but the syllable counts differ.~\cite{UvodTeorieVerse}

\begin{figure}[htpb]
    \centering
    \settowidth{\versewidth}{A | \uline{vrá}|til | se | \uline{Mu}|ro|mec | \uline{k dob}|ré|mu | \uline{mlád}|ci, \textsubscript{(4 acc., 12 syll.)}}
    \begin{verse}[\versewidth]
    A | \uline{vrá}|til | se | \uline{Mu}|ro|mec | \uline{k dob}|ré|mu | \uline{mlád}|ci, \textsubscript{(4 acc., 12 syll.)}\\
    \uline{K mlád}|ci | \uline{to}|mu | \uline{dob}|ré|mu, | \uline{u}|bi|té|mu; \textsubscript{(4 acc., 11 syll.)}\\
    On | \uline{vy}|ko|pal | \uline{hrob} | \uline{v ší}|rém | \uline{po}|li, \textsubscript{(4 acc., 9 syll.)}\\
    \uline{Do} | to|ho | \uline{hro}|bu | \uline{tě}|lo | \uline{po}|lo|žil \textsubscript{(4 acc., 10 syll.)}\\
    \end{verse}
    \caption[Tonic versification]{Tonic versification (accented syllables are underlined)}\label{fig:tonic-versification}
\end{figure}

\subsection{Syllabotonic versification system}
Syllabotonic versification combines syllabic and tonic versification considering not only the number of syllables but also whether they are accented or not.~\cite{UvodTeorieVerse}

In the excerpt from the poem \emph{U studánky} by Jan Neruda (see Figure~\ref{fig:syllabotonic-versification}), every accented syllable, except the second syllable in the fourth verse, occupies an odd position in verse. Furthermore, every verse has exactly eight syllables. Therefore, syllabotonic versification is used.~\cite{TeorieLiteraturySS}

\begin{figure}[htpb]
    \centering
    \settowidth{\versewidth}{\uline{ble}|dé | \uline{ja}|ko | \uline{ru}|báš | \uline{z kmen}|tu. \textsubscript{(8 syll.)}}
    \begin{verse}[\versewidth]
    \uline{U} | stu|dá|nky | \uline{sto}|jí | \uline{děv}|če, \textsubscript{(8 syll.)}\\
    \uline{mla}|dé | \uline{ja}|ko | \uline{strů}|mek | \uline{mla}|dý, \textsubscript{(8 syll.)}\\
    \uline{ble}|dé | \uline{ja}|ko | \uline{ru}|báš | \uline{z kmen}|tu. \textsubscript{(8 syll.)}\\
    A | \uline{na} | ne|bi | \uline{bí}|lý | \uline{mě}|síc, \textsubscript{(8 syll.)}\\
    \uline{ko}|lem | \uline{ně}|ho | \uline{vod}|ní | \uline{ko}|lo \textsubscript{(8 syll.)}\\
    \uline{jak} | by | \uline{ze} | stu|dá|nky | \uline{hle}|děl. \textsubscript{(8 syll.)}\\
    \end{verse}
    \caption[Syllabotonic versification]{Syllabotonic versification (accented syllables are underlined)}\label{fig:syllabotonic-versification}
\end{figure}

In the rest of this work, the Czech syllabotonic verse will be discussed.

\section{Metrical analysis properties}
When performing a metrical analysis of a poem, various properties of the verse can be examined.

\subsection{Foot}
The basic metrical unit of a verse is called a foot. In the syllabotonic verse, it represents a group of at least two syllables that is repeated regularly throughout the verse. One foot consists of strong and weak positions. Strong positions are labelled with \textbf{S}, and weak positions are labelled with \textbf{W}. If there are two weak positions within a foot, the first is labelled using \textbf{V}. Table~\ref{tab:czech-syllabotonic-feet} presents all types of feet that can be encountered within the Czech syllabotonic verse.~\cite{UvodTeorieVerse}

\begin{table}[htpb]
\caption[Czech syllabotonic verse feet]{Czech syllabotonic verse feet (positions inside brackets can be omitted)}\label{tab:czech-syllabotonic-feet}
\centering
\begin{tabular}{|c||l|}\hline
    Foot & Feet pattern\\\hline\hline
    Iamb & W\textsubscript{0} S\textsubscript{1} W\textsubscript{1} S\textsubscript{2} \ldots{} S\textsubscript{n} (W\textsubscript{n})\\
    Trochee & S\textsubscript{1} W\textsubscript{1} S\textsubscript{2} W\textsubscript{2} \ldots{} S\textsubscript{n} (W\textsubscript{n})\\
    Dactyl & S\textsubscript{1} V\textsubscript{1} W\textsubscript{1} S\textsubscript{2} V\textsubscript{2} W\textsubscript{2} \ldots{} S\textsubscript{n} ((V\textsubscript{n}) W\textsubscript{n})\\
    Dactyl with anacrusis (Amphibrach) & W\textsubscript{0} S\textsubscript{1} V\textsubscript{1} W\textsubscript{1} S\textsubscript{2} V\textsubscript{2} W\textsubscript{2} \ldots{} S\textsubscript{n} ((V\textsubscript{n}) W\textsubscript{n})\\
    Dactylotrochee & S\textsubscript{1} (V\textsubscript{1}) W\textsubscript{1} S\textsubscript{2} (V\textsubscript{2}) W\textsubscript{2} \ldots{} S\textsubscript{n} ((V\textsubscript{n}) W\textsubscript{n})\\
    Dactylotrochee with anacrusis & W\textsubscript{0} S\textsubscript{1} (V\textsubscript{1}) W\textsubscript{1} S\textsubscript{2} (V\textsubscript{2}) W\textsubscript{2} \ldots{} S\textsubscript{n} ((V\textsubscript{n}) W\textsubscript{n})\\\hline
\end{tabular}
\end{table}

All standard syllabotonic metrical patterns can be expressed by the following regular expression:

\begin{equation}\label{eq:syllabotonic-regex}
        \verb|^W?(SWW?)*(SW?)?$|
\end{equation}

where V and W weak positions are annotated with the same symbol. \cite{KVETA}

It is important to note that foot and word are two different concepts. Their boundaries do not have to overlap. Two different situations are distinguished:
\begin{description}
\item[Caesura] The word does not end where the foot ends.
\item[Diaeresis] The word ends with a foot end.~\cite{UvodTeorieVerse}
\end{description}

The example of a poetic text written in quantitative iamb (see Figure~\ref{fig:caesura-diaeresis}) might help clarify both definitions. There, a diaeresis can be found, for example, after the words \emph{vidět} and \emph{není} in the second verse. Caesura occurs, for example, after the word \emph{hrozno} in the first verse or the word \emph{milence} in the fourth verse.~\cite{TeorieLiteraturySS}

\begin{figure}[htpb]
    \centering
    \settowidth{\versewidth}{Tma | \uline{jest} | a | \uline{hroz}|no | \uline{vů}|kol,}
    \begin{verse}[\versewidth]
    Tma | \uline{jest} | a | \uline{hroz}|no | \uline{vů}|kol,\\
    vi|\uline{dět} | ne|\uline{ní} | sle|\uline{dů};\\
    kte|\uline{rá} | a|\uline{si} | ste|\uline{zi}|čka\\
    ve|\uline{de} | k mi|\uline{len}|ce | \uline{mé}?\\
    \end{verse}
    \caption[Caesura and diaeresis]{Caesura and diaeresis (long syllables are underlined)}\label{fig:caesura-diaeresis}
\end{figure}

\subsection{Metre}
The repetition of metrical feet in a verse forms a metre -- the abstract outline of a verse.~\cite{UvodTeorieVerse}

\subsection{Rhythm versus metre}
The main complexity of the metre assignment task lies in the difference between a rhythm and a~metre. When talking about the syllabotonic verse, metre is expressed by the regular alternation of strong and weak positions. On the other hand, rhythm is the poet's actual implementation of the metre using the alternation of accented and non-accented syllables.

For the syllabotonic verse, the underlying concept is that S-positions correspond to accented syllables and V-positions and W-positions to non-accented ones. However, in reality, all positions can correspond to both accented and non-accented ones. In many situations, the poet has the freedom to choose whether to use an accent. As a result, one metre can be expressed by multiple rhythmical patterns.

For the Czech syllabotonic verse, there exist complex rules determining in which situations it is possible to use accented or non-accented syllables. The rules were obtained through a thorough analysis of many poems. Naturally, these rules do not necessarily cover all poems that have ever existed. Sometimes a poem that violates them can be encountered.~\cite{UvodTeorieVerse}

\subsection{Clause (Line ending)}
The ending of a verse is called a clause. In the syllabotonic verse, three types of clauses are distinguished based on the last position of a verse:
\begin{itemize}
\item masculine,
\item feminine,
\item acatalectic.
\end{itemize}

Verses with masculine endings end with the S-position. When the verse ends with W-position, it can either be feminine or acatalectic. The acatalectic verses end with the SVW position pattern, and the feminine verses with the SW pattern.~\cite{UvodTeorieVerse} Moreover, as acatalectic are also annotated verses that end with the SV pattern.~\cite{GitCorpusCzechVerse}

\subsection{Verse multimetry and poem polymetry}
A verse is labelled multimetric when its rhythmical pattern can correspond to more metres. The correct metre of such a verse is then selected based on the surrounding context.

A similar concept to multimetry is polymetry, but this time regarding a whole poem. A poem is considered polymetric when some of its verses have different metres assigned than others, and the occurrences of such metres are more or less predictable.~\cite{UvodTeorieVerse}

\subsection{Metrical tagging example}
The metrically tagged Czech syllabotonic poetic text (see Figure~\ref{fig:syllabotonic-metrical-tagging}) illustrates some of the presented verse properties:

\begin{description}
\item[First verse] Dactyl with four feet and a masculine clause.
\item[Second verse] Dactyl with three feet and an acatalectic clause.
\item[Third verse] Dactyl with anacrusis with three feet and a feminine clause.
\item[Fourth verse] Dactylotrochee with anacrusis with three feet and a feminine clause. Although accented, the first syllable of the fourth verse represents a weak position.~\cite{UvodTeorieVerse}
\end{description}

\begin{figure}[htpb]
    \centering
    \settowidth{\versewidth}{\uline{Pr\sS{a}}|m\sV{é}|n\sW{e}k | \uline{z\sS{a}z}|v\sV{o}|n\sW{i}l | \uline{t\sS{i}}|š\sV{e} | \sW{a} | \uline{r\sS{á}d}. \textsubscript{(D4m)}}
    \begin{verse}[\versewidth]
    \uline{Pr\sS{a}}|m\sV{é}|n\sW{e}k | \uline{z\sS{a}z}|v\sV{o}|n\sW{i}l | \uline{t\sS{i}}|š\sV{e} | \sW{a} | \uline{r\sS{á}d}. \textsubscript{(D4m)}\\
    \uline{V s\sS{r}d}|c\sV{i} | m\sW{é}m | \uline{p\sS{o}z}|d\sV{i}l | s\sW{e} | \uline{l\sS{i}s}|t\sV{o}|p\sW{a}d \textsubscript{(D3a)}\\
    \sW{a} | \uline{st\sS{u}}|d\sV{u}|j\sW{u} | \uline{vl\sS{a}}|stn\sV{í} | sv\sW{é} | \uline{r\sS{y}}|s\sW{y} \textsubscript{(Da3f)}\\
    \uline{j\sW{á}} | \uline{z\sS{a}}|p\sV{o}|mn\sW{ě}l | \uline{\sS{u}m}|ř\sW{í}t | \uline{kd\sS{y}}|s\sW{i}. \textsubscript{(DTa3f)}
    \end{verse}
    \caption[Syllabotonic metrical tagging]{Syllabotonic metrical tagging (Accented syllables are underlined. Strong and weak positions and line tags containing metre, number of feet, and clause are annotated.)}\label{fig:syllabotonic-metrical-tagging}
\end{figure}

\section{Special types of verse}
In addition to standard metres (see Table~\ref{tab:czech-syllabotonic-feet}), some special types of verse can also be found in the Czech syllabotonic tradition. Some of them, which are discussed further in the thesis, are presented.

\subsection{Imitations of hexametre, pentametre, elegiac couplet}
\subsubsection{Hexametre}
Hexametre originally comes from ancient Greek poetry, where it was one of the most widely used metres. Later, it was adopted by the Romans and, from them, spread to medieval Europe. It represents a quantitative dactyl consisting of six feet. An important element of the hexametre verse is a caesura. In the Czech syllabotonic tradition, hexametre imitations began to appear in the 19\textsuperscript{th} century during the Czech National Revival.~\cite{UvodTeorieVerse} In the syllabotonic hexametre every line contains 12 to 17 syllables, and its metrical pattern must match the following regular expression (V and W weak positions are annotated with the same symbol):

\begin{equation}\label{eq:hexametre-regex}
        \verb|^SWW?SWW?SWW?SWW?SWW?SW$|.\text{ \cite{KVETA}}
\end{equation}

For one of the syllabotonic hexametre poems tagged inside the Corpus of Czech Verse -- \emph{Komu platí přízvuk.} by František Vladislav Hek -- see Figure~\ref{fig:syllabotonic-hexametre}.~\cite{GitCorpusCzechVerse}

\begin{figure}[htpb]
    \centering
    \settowidth{\versewidth}{\uline{Př\sS{í}}|zv\sW{u}k | \uline{pr\sS{a}}|v\sV{i}|dl\sW{e}m | j\sS{e}st – | kd\sW{y}ž | \uline{h\sS{o}}|d\sV{í} | s\sW{e} | \uline{n\sS{a}} | p\sV{r}s|t\sW{y} | \uline{v\sS{o}l}|n\sW{ě}. \textsubscript{(6f)}}
    \begin{verse}[\versewidth]
    \uline{St\sS{a}}|t\sV{e}č|n\sW{ý} | \uline{\sS{A}}|g\sW{a}|m\sS{e}|mn\sV{o}n | j\sW{a}k | \uline{r\sS{y}ch}|l\sV{e} | s\sW{e} | \uline{d\sS{o}} | č\sV{e}s|k\sW{ý}ch | \uline{b\sS{á}s}|n\sW{í} \textsubscript{(6f)}\\
    \uline{d\sS{o}s}|t\sV{a}l, | hn\sW{e}d | \uline{j\sS{e}}|h\sW{o} | jm\sS{ě} | js\sW{o}u | \uline{n\sS{a}} | p\sV{r}s|t\sW{y} | \uline{m\sS{ě}}|ř\sV{i}|l\sW{i}. | \uline{J\sS{e}d}|n\sV{i} \textsubscript{(6a)}\\
    \uline{v\sS{á}}|ž\sV{í}|c\sW{e} | \uline{ctn\sS{o}st} | t\sW{a}k | \uline{sl\sS{a}v}|n\sV{é}|h\sW{o} | \uline{h\sS{r}}|d\sV{i}|n\sW{y} – | \uline{př\sS{í}z}|v\sV{u}č|n\sW{ě} | \uline{ps\sS{a}}|l\sW{i}; \textsubscript{(6f)}\\
    \uline{pr\sS{o}} | p\sV{o}|h\sW{o}d|l\sS{í} | vš\sW{a}k | sv\sS{é}, | jm\sW{ě} | \uline{n\sS{a}} | pr\sV{a}|h\sW{u} | \uline{zkr\sS{á}}|t\sV{i}|l\sW{i} | \uline{dr\sS{u}}|z\sV{í}. \textsubscript{(6a)}\\
    \uline{Př\sS{í}}|zv\sW{u}k | \uline{pr\sS{a}}|v\sV{i}|dl\sW{e}m | j\sS{e}st – | kd\sW{y}ž | \uline{h\sS{o}}|d\sV{í} | s\sW{e} | \uline{n\sS{a}} | p\sV{r}s|t\sW{y} | \uline{v\sS{o}l}|n\sW{ě}. \textsubscript{(6f)}\\
    \end{verse}
    \caption[Syllabotonic hexametre]{Syllabotonic hexametre (Accented syllables are underlined. Strong and weak positions and line tags containing number of feet and clause are annotated.)}\label{fig:syllabotonic-hexametre}
\end{figure}

\subsubsection{Pentametre}
The pentametre contains, perhaps surprisingly, not five, but again six dactylic feet. It was rarely used alone; instead, it was used in combination with hexametre inside an elegiac couplet.~\cite{UvodTeorieVerse} In the syllabotonic pentametre, every line must contain 10 to 15 syllables, and its metrical pattern must match the pentametre regular expression (V and W weak positions are annotated with the same symbol):

\begin{equation}\label{eq:pentametre-regex}
        \verb|^SWW?SWW?SW?SWW?SWW?S$|.\text{ \cite{KVETA}}
\end{equation}

For an example of a syllabotonic pentametre poem  annotated within the Corpus of Czech Verse -- \emph{PODZIM V PARKU} by Jaroslav Vrchlický -- see Figure~\ref{fig:syllabotonic-pentametre}.~\cite{GitCorpusCzechVerse}

\begin{figure}[htpb]
    \centering
    \settowidth{\versewidth}{\uline{tl\sS{u}}|m\sV{o}|č\sW{í}š | \uline{v\sS{ě}r}|n\sW{ě}, | \uline{m\sS{l}č}|k\sW{y} | \sS{a} | \uline{v\sV{á}ž}|n\sW{ě}: | „\uline{Vš\sS{e}}|ck\sV{o} | j\sW{e}st | \uline{d\sS{ý}m}!“ \textsubscript{(6m)}}
    \begin{verse}[\versewidth]
    \uline{M\sS{o}}|d\sV{e}r|n\sW{í}ch | \uline{č\sS{i}}|v\sV{ů} | t\sW{y}’s | \uline{v hn\sS{ě}}|d\sW{é} | \uline{k\sS{á}}|p\sW{i} | \uline{zp\sS{o}}|v\sV{ě}d|n\sW{í}k, | \uline{v\sS{í}m}, \textsubscript{(6m)}\\
    \uline{tl\sS{u}}|m\sV{o}|č\sW{í}š | \uline{v\sS{ě}r}|n\sW{ě}, | \uline{m\sS{l}č}|k\sW{y} | \sS{a} | \uline{v\sV{á}ž}|n\sW{ě}: | „\uline{Vš\sS{e}}|ck\sV{o} | j\sW{e}st | \uline{d\sS{ý}m}!“ \textsubscript{(6m)}
    \end{verse}
    \caption[Syllabotonic pentametre]{Syllabotonic pentametre (Accented syllables are underlined. Strong and weak positions and line tags containing number of feet and clause are annotated.)}\label{fig:syllabotonic-pentametre}
\end{figure}

\subsubsection{Elegiac couplet}
In the syllabotonic elegiac couplet, the hexametres and pentametres alternate regularly. All odd lines correspond to the metrical pattern of the hexameter, and all even lines correspond to the metrical pattern of the pentametre.~\cite{KVETA}

\subsection{Ghazal poems}
Ghazals are poems in which the first and every even line contains a so-called \emph{radif} -- repeating word or a group of words at the end of the line. The lines containing the radif are then assigned a combination of two different metrical patterns, one pattern for the part without the radif and one pattern for the part containing the radif. Therefore, the resulting metric pattern does not need to correspond to any standard syllabotonic metre.~\cite{KVETA}

In the ghazal poem \emph{Vavřín} by Jaroslav Vrchlický (see Figure~\ref{fig:ghazal}), which is annotated within the Corpus of Czech verse, the radif is represented by the word \emph{vavřín}. The part without the radif corresponds to the trochaic metrical pattern.~\cite{GitCorpusCzechVerse} When concatenated with the metrical pattern of the radif part, the resulting metrical pattern does not correspond to any standard syllabotonic metre, as two strong positions next to each other are not allowed (see regular expression~\eqref{eq:syllabotonic-regex}).

\begin{figure}[htpb]
    \centering
    \settowidth{\versewidth}{\uline{l\sS{e}h}|k\sW{ý}, | \uline{ž\sS{e}} | j\sW{í}m | \uline{v\sS{í}}|t\sW{r} | \uline{hr\sS{a}}|v\sW{ě} | \uline{zm\sS{í}}|t\sW{á} | s\sS{e}m | \uline{\sW{a}} | t\sS{a}m, \textsubscript{(7m)}}
    \begin{verse}[\versewidth]
    \uline{Tm\sS{a}}|v\sW{é}, | \uline{sm\sS{u}t}|n\sW{é} | \uline{l\sS{í}st}|k\sW{y} | \uline{v\sS{y}}|h\sW{á}|n\sS{í} | \uline{v\sS{a}v}|ř\sW{í}n, \textsubscript{(6f)}\\
    \uline{h\sS{o}ř}|k\sW{o}u | \uline{v\sS{ů}}|n\sW{í} | \uline{t\sS{r}p}|c\sW{e} | \uline{z\sS{a}}|v\sW{á}|n\sS{í} | \uline{v\sS{a}v}|ř\sW{í}n, \textsubscript{(6f)}\\
    \uline{l\sS{e}h}|k\sW{ý}, | \uline{ž\sS{e}} | j\sW{í}m | \uline{v\sS{í}}|t\sW{r} | \uline{hr\sS{a}}|v\sW{ě} | \uline{zm\sS{í}}|t\sW{á} | s\sS{e}m | \uline{\sW{a}} | t\sS{a}m, \textsubscript{(7m)}\\
    \uline{b\sS{a}l}|v\sW{a}|n\sS{y} | j\sW{e} | \uline{t\sS{ě}ž}|k\sW{ý} | \uline{n\sS{a}} | skr\sW{á}|n\sS{i} | \uline{v\sS{a}v}|ř\sW{í}n. \textsubscript{(6f)}\\
    \uline{F\sS{i}r}|d\sW{u}|s\sS{i}|h\sW{o} | \uline{hr\sS{o}b} | j\sW{e}n | \uline{p\sS{o}}|sl\sW{i} | \uline{š\sS{a}}|ch\sW{a} | \uline{n\sS{a}j}|d\sW{o}u, \textsubscript{(6f)}\\
    \uline{m\sS{i}}|l\sW{u}|j\sS{e}ť | vžd\sW{y} | \uline{t\sS{a}}|k\sW{é} | \uline{s\sS{e}t}|k\sW{á}|n\sS{í} | \uline{v\sS{a}v}|ř\sW{í}n. \textsubscript{(6f)}\\
    \end{verse}
    \caption[Ghazal poem]{Ghazal poem (Accented syllables are underlined. Strong and weak positions and line tags containing number of feet and clause are annotated.)}\label{fig:ghazal}
\end{figure}